\section{Fórmulas}

\PN \textbf{FÓRMULAS DE TIPO $\tau$:}
	\begin{eqnarray*}
		F_{0}^{\tau} &=& \{(t \equiv s): t, s \in T^{\tau}\} \cup \{r(t_{1}, \dotsc, t_{n}): r \in \mathcal{R}_{n}, n \geq
			1, t_{1}, \dotsc, t_{n} \in T^{\tau}\} \\
		F_{k+1}^{\tau} &=& F_{k}^{\tau} \cup \{\lnot \varphi: \varphi \in F_{k}^{\tau}\} \cup \{(\varphi \vee \psi):
			\varphi, \psi \in F_{k}^{\tau}\} \cup \{(\varphi \wedge \psi): \varphi, \psi \in F_{k}^{\tau}\} \cup \\
		&& \{(\varphi \rightarrow \psi): \varphi, \psi \in F_{k}^{\tau}\} \cup \{(\varphi \leftrightarrow \psi): \varphi,
			\psi \in F_{k}^{\tau}\} \cup \{\forall v \varphi: \varphi \in F_{k}^{\tau}, v \in Var\} \cup \\
		&& \{\exists v\varphi :\varphi \in F_{k}^{\tau}, v \in Var\} \\
		\\
		F^{\tau} &=& \bigcup_{k \geq  0} F_{k}^{\tau}
	\end{eqnarray*}

	\vspace{3mm}
	\PN \textbf{SUBFORMULAS:}
  \PN Una fórmula $\varphi$ será llamada \textbf{subfórmula} (propia) de una fórmula $\psi$, cuando $\varphi$ sea no
  igual a $\psi$ y tenga alguna ocurrencia en $\psi$.
