\documentclass[12pt,a4paper]{article}
\usepackage[utf8]{inputenc}[spanish]
\usepackage{amsmath}
\usepackage{amsfonts}
\usepackage{amssymb}
\usepackage{lmodern}
\usepackage{amsmath}
\usepackage{amsthm}
\usepackage{enumerate}
\usepackage{graphicx}
\usepackage{mathtools}
\usepackage{stackrel}
\usepackage[left=2cm,right=2cm,top=2cm,bottom=2cm]{geometry}
\newcounter{neq}
\providecommand{\abs}[1]{\lvert#1\rvert}
\newcommand{\SIGMA}{\Sigma^{\ast}}
\newcommand{\PN}{\par\noindent}
\newcommand{\SU}{\mathsf{s}}
\newcommand{\IN}{\mathsf{i}}

\title{Resumen Física}

\begin{document}
	\begin{center}
		\Huge \textbf{Definiciones de Lógica 2017} \\
		\vspace{3mm}
		\large Agustín Curto
	\end{center}
	
	\section{Posets}
  \PN \textbf{ORDEN PARCIAL:} Sea $P \neq \emptyset$ cualquiera, una relación binaria $\leq$ sobre $ P$ será llamada un
  \textbf{orden parcial} sobre $P$ si se cumplen las siguientes condiciones:
  \begin{enumerate}
    \item $\leq$ es \textbf{reflexiva}, i.e $a \leq a \ \ \forall a \in P$.
    \item $\leq$ es \textbf{antisimétrica}, i.e si $a \leq b$ y $b \leq a \Rightarrow a=b \ \ \forall a, b \in P$.
    \item $\leq$ es \textbf{transitiva}, i.e si $a \leq b$ y $b \leq c \Rightarrow a \leq c \ \ \forall a, b, c \in P$.
  \end{enumerate}

  \PN \textbf{POSET:} Un conjunto parcialmente ordenado o \textbf{poset} será un par $(P,\leq)$ donde:
  \begin{itemize}
    \item $P \neq \emptyset$ cualquiera
    \item $\leq$ es un orden parcial sobre $P$
  \end{itemize}

  \PN \textbf{RELACIONES BINARIAS $<, \prec$:} Dado un poset $(P, \leq)$ definimos $, \prec$ sobre $P$ de la siguiente
  manera:
  \begin{eqnarray*}
    a < b &\Leftrightarrow& a \leq b \text{ y } a \neq b \\
    a \prec b &\Leftrightarrow& a < b \text{ y } \nexists z \ a < z < b
  \end{eqnarray*}

  \PN \textbf{DEFINICIONES:} Sea $(P,\leq)$ un poset, entonces:
  \begin{itemize}
    \item \textbf{Maximal:} $a \in P$ es un elemento maximal de $(P,\leq)$ si $a \nless b, \ \forall b \in P$.
    \item \textbf{Minimal:} $a \in P$ es un elemento minimal de $(P,\leq)$ si $b \nless a, \ \forall b \in P$.
    \item \textbf{Máximo:} $a \in P$ es el elemento máximo de $(P,\leq)$ si $b \leq a, \ \forall b \in P$.
    \item \textbf{Mínimo:} $a \in P$ es el elemento mínimo de $(P,\leq )$ si $a \leq b, \ \forall b \in P$.

    \vspace{3mm}
    \PN Dado $S \subseteq P$:
    \item \textbf{Cota superior:} $a \in P$ es cota superior de $S$ en $(P,\leq)$ cuando $b \leq a, \ \forall b \in S$.
    \item \textbf{Cota inferior:} $a \in P$ es cota inferior de $S$ en $(P,\leq)$ cuando $a \leq b, \ \forall b \in S$
    \item \textbf{Supremo:} $a \in P$ será llamado supremo de $S$ en $(P,\leq)$ cuando se den las siguientes
    condiciones:
      \begin{enumerate}
        \item $a$ es a cota superior de $S$ en $(P,\leq)$
        \item Para cada $b \in P$, si $b$ es una cota superior de $S$ en $ (P,\leq) \Rightarrow a \leq b$.
      \end{enumerate}
    \item \textbf{Ínfimo:} $a \in P$ será llamado ínfimo de $S$ en $(P,\leq)$ cuando se den las siguientes
    condiciones:
      \begin{enumerate}
        \item $a$ es a cota inferior de $S$ en $(P,\leq)$
        \item Para cada $b \in P$, si $b$ es una cota inferior de $S$ en $ (P,\leq) \Rightarrow b \leq a$.
      \end{enumerate}
  \end{itemize}

  \PN \textbf{HOMOMORFISMO E ISOMORFISMO DE POSETS:} Sean $(P,\leq)$ y $(P^{\prime},\leq^{\prime})$ posets
  \begin{itemize}
    \item Una función $F: P \rightarrow P^{\prime}$ será llamada un \textbf{homomorfismo} de $(P,\leq)$ en
      $(P^{\prime},\leq^{\prime})$ si $ \forall x, y \in P$ se cumple que $x \leq y \Rightarrow F(x) \leq^{\prime} F(y)$.
    \item Una función $F: P \rightarrow P^{\prime}$ será llamada un \textbf{isomorfismo} de $(P,\leq)$ en
      $(P^{\prime},\leq^{\prime})$ si $F$ es \textbf{biyectiva} y tanto $F$ como $F^{-1}$ son \textbf{homomorfismos}.
  \end{itemize}

	\section{Reticulados}

  \PN \textbf{RETICULADO:}
  \begin{enumerate}
    \item Un conjunto parcialmente ordenado $(L,\leq)$ es un \textbf{reticulado} si $\forall a, b \in L$, existen
    $\sup(\{a,b\})$ e $\inf(\{a,b\})$. Se definen:
    \[
      \begin{array}{rcl}
        a \ \SU \ b &=& \sup(\{a,b\}) \\
        a \ \IN \ b &=& \inf(\{a,b\})
      \end{array}
    \]

    \item Una terna $(L, \SU,\IN)$, donde $L \neq \emptyset$ cualquiera, $x, y, z \in L$ cualquieras y $\SU$ e
    $\IN$ son dos operaciones binarias sobre $L$ será llamada \textbf{reticulado} cuando cumpla las siguientes
    identidades:
    \begin{enumerate}
      \item[(I1)] $x \ \SU \ x = x \ \IN \ x = x$
      \item[(I2)] $x \ \SU \ y = y \ \SU \ x$
      \item[(I3)] $x \ \IN \ y = y \ \IN \ x$
      \item[(I4)] $(x \ \SU \ y) \ \SU \ z = x \ \SU \ (y \ \SU \ z)$
      \item[(I5)] $(x \ \IN \ y) \ \IN \ z = x \ \IN \ (y \ \IN \ z)$
      \item[(I6)] $x \ \SU \ (x \ \IN \ y) = x$
      \item[(I7)] $x \ \IN \ (x \ \SU \ y) = x$
    \end{enumerate}
  \end{enumerate}

  \vspace{3mm}
  \PN \textbf{SUBRETICULADO:} Sea $(L, \SU, \IN)$ un reticulado. $S \neq \emptyset \subseteq L$ será
  llamado \textbf{subuniverso} de $(L, \SU, \IN)$ si es cerrado bajo las operaciones $\SU$ e $\IN$. Diremos que el
  reticulado $(S, \SU\mathrm{\mid}_{S \times S}, \IN\mathrm{\mid}_{S \times S})$ es \textbf{subreticulado} de
  $(L, \SU, \IN)$.

  \vspace{3mm}
  \PN \textbf{HOMOMORFISMO E ISOMORFISMO DE RETICULADOS:} Sean $(L, \SU, \IN)$ y $(L^{\prime}, \SU^{\prime},
  \IN^{\prime})$ reticulados.
  \begin{itemize}
    \item Una función $F: L \rightarrow L^{\prime}$ será llamada un \textbf{homomorfismo} de $(L, \SU, \IN)$ en
      $(L^{\prime}, \SU^{\prime}, \IN^{\prime})$ si $\forall x, y \in L$ se cumple que:
      \[
        \begin{array}{rcl}
          F(x \mathsf{\;s\;} y) &=& F(x) \ \SU^{\prime} \ F(y) \\
          F(x \mathsf{\;i\;} y) &=& F(x) \ \IN^{\prime} \ F(y)
        \end{array}
      \]
    \item Una función $F: L \rightarrow L^{\prime}$ será llamada un \textbf{isomorfismo} de $(L, \SU, \IN)$ en
      $(L^{\prime}, \SU^{\prime}, \IN^{\prime})$ si $F$ es \textbf{biyectiva} y tanto $F$ como $F^{-1}$ son
      \textbf{homomorfismos}.
  \end{itemize}

  \vspace{3mm}
  \PN \textbf{CONGRUENCIAS DE RETICULADOS:} Sea $(L, \SU, \IN)$ un reticulado, una \textbf{congruencia} sobre
  $(L, \SU, \IN)$ será una \textbf{relación de equivalencia} $\theta$ la cual cumpla:
  \[
    x \theta x^{\prime} \text{ y } y \theta y^{\prime} \Rightarrow (x \ \SU \ y) \theta (x^{\prime} \ \SU \ y^{\prime})
    \text{ y } (x \ \IN \ y) \theta (x^{\prime} \ \IN \ y^{\prime})
  \]

  \PN Definimos, sobre $L/\theta$, $\mathsf{\tilde{s}}$ e $\mathsf{\tilde{\imath}}$, de la siguiente manera:
  \[
    \begin{array}{rcl}
      x/\theta \ \mathsf{\tilde{s}} \ y/\theta &=& (x \ \SU \ y)/\theta \\
      x/\theta \ \mathsf{\tilde{i}} \ y/\theta &=& (x \ \IN \ y)/\theta
    \end{array}
  \]

  \vspace{3mm}
  \PN \textbf{KERNEL:} Dada una función $F: A \rightarrow B$, llamaremos núcleo de $F$ a la relación binaria:
  \[
    \{(a,b) \in A^{2}: F(a) = F(b)\}
  \]

  \PN \textbf{Notación:} $\ker F$.

  \vspace{3mm}
  \PN \textbf{PROYECCIÓN CANÓNICA:} Si $R$ es una \textbf{relación de equivalencia} sobre un conjunto $A$, definimos la
  función:
  \[
    \begin{array}{ccc}
      \pi_{R}: && A \rightarrow A/R \\
      && a \rightarrow a/R
    \end{array}
  \]

	\section{Reticulados Acotados}

  \PN \textbf{RETICULADO ACOTADO:} Sea $(L, \SU, \IN, 0, 1)$, donde $L \neq \emptyset, \ \SU$ e $\IN$ operaciones
  binarias sobre $L$ y $0, 1 \in L$, será llamada un \textbf{reticulado acotado} si $(L, \SU, \IN)$ es un reticulado y
  además se cumplen las siguientes identidades:
  \begin{enumerate}
    \item[(I8)] $0 \ \SU \ x = x$, para cada $x \in L$
    \item[(I9)] $x \ \SU \ 1 = 1$, para cada $x \in L$
  \end{enumerate}

  \vspace{3mm}
  \PN \textbf{SUBRETICULADO ACOTADO:} Sea $(L, \SU, \IN, 0, 1)$ un reticulado acotado. $S \neq \emptyset \subseteq L$
  será llamado \textbf{subuniverso} de $(L, \SU, \IN, 0, 1)$ si $0, 1 \in S$ y es cerrado bajo las operaciones $\SU$ e
  $\IN$. Diremos que el reticulado acotado $(S, \SU\mathrm{\mid}_{S \times S}, \IN\mathrm{\mid}_{S \times S}, 0, 1)$ es
  \textbf{subreticulado acotado} de $(L, \SU, \IN, 0, 1)$.

  \vspace{3mm}
  \PN \textbf{HOMOMORFISMO E ISOMORFISMO DE RETICULADOS ACOTADOS:} Sean $(L, \SU, \IN, 0, 1)$ y
  $(L^{\prime}, \SU^{\prime}, \IN^{\prime}, 0^{\prime}, 1^{\prime})$ reticulados acotados.
  \begin{itemize}
    \item Una función $F: L \rightarrow L^{\prime}$ será llamada un \textbf{homomorfismo} de $(L, \SU, \IN, 0, 1)$ en
      $(L^{\prime}, \SU^{\prime}, \IN^{\prime}, 0^{\prime}, 1^{\prime})$ si $\forall x, y \in L$ se cumple que:
      \[
        \begin{array}{rcl}
          F(x \mathsf{\;s\;} y) &=& F(x) \ \SU^{\prime} \ F(y) \\
          F(x \mathsf{\;i\;} y) &=& F(x) \ \IN^{\prime} \ F(y) \\
          F(0) &=& 0^{\prime} \\
          F(1) &=& 1^{\prime}
        \end{array}
      \]
    \item Una función $F: L \rightarrow L^{\prime}$ será llamada un \textbf{isomorfismo} de $(L, \SU, \IN, 0, 1)$ en
      $(L^{\prime}, \SU^{\prime}, \IN^{\prime}, 0^{\prime}, 1^{\prime})$ si $F$ es \textbf{biyectiva} y tanto $F$ como
      $F^{-1}$ son \textbf{homomorfismos}.
  \end{itemize}

  \vspace{3mm}
  \PN \textbf{CONGRUENCIAS DE RETICULADOS ACOTADOS:} Sea $(L, \SU, \IN, 0, 1)$ un reticulado, una \textbf{congruencia}
  sobre $(L, \SU, \IN, 0, 1)$ será una \textbf{relación de equivalencia} $\theta$ la cual cumpla:
  \[
    x \theta x^{\prime} \text{ y } y \theta y^{\prime} \Rightarrow (x \ \SU \ y) \theta (x^{\prime} \ \SU \ y^{\prime})
    \text{ y } (x \ \IN \ y) \theta (x^{\prime} \ \IN \ y^{\prime})
  \]

  \PN Definimos, sobre $L/\theta$, $\mathsf{\tilde{s}}$ e $\mathsf{\tilde{\imath}}$, de la siguiente manera:
  \[
    \begin{array}{rcl}
      x/\theta \ \mathsf{\tilde{s}} \ y/\theta &=& (x \ \SU \ y)/\theta \\
      x/\theta \ \mathsf{\tilde{i}} \ y/\theta &=& (x \ \IN \ y)/\theta
    \end{array}
  \]

	\section{Reticulados Complementados}

  \PN \textbf{COMPLEMENTO:} Sea $(L,\mathsf{s},\mathsf{i},0,1)$ un reticulado acotado. Dado $a \in L$, diremos que
  $a$ es \textbf{complementado}, si $\exists b \in L$ tal que:
  \[
    \begin{array}{rcl}
      a \ \SU \ b &=& 1 \\
      a \ \IN \ b &=& 0
    \end{array}
  \]

  \PN \textbf{RETICULADO COMPLEMENTADO:} $(L, \SU, \IN, ^{c}, 0, 1)$, donde $L \neq \emptyset, \SU$ e $\IN$ son
  operaciones binarias sobre $L$, $^{c}$ es una operación unaria sobre $L$ y $0, 1 \in L$, será llamada un
  \textbf{reticulado complementado} si $(L, \SU, \IN, 0, 1)$ es un reticulado acotado y además:
  \begin{enumerate}
    \item[(I10)] $x \ \SU \ x^{c} = 1$, para cada $x \in L$
    \item[(I11)] $x \ \IN \ x^{c} = 0$, para cada $x \in L$
  \end{enumerate}

  \vspace{3mm}
  \PN \textbf{SUBRETICULADO COMPLEMENTADO:} Dado $(L, \SU, \IN, ^{c}, 0, 1)$ reticulado complementado, diremos que
  $(L^{\prime}, \SU^{\prime}, \IN^{\prime}), ^{c^{\prime}}, 0^{\prime}, 1^{\prime}$ es \textbf{subreticulado
  complementado} de $(L, \SU, \IN, ^{c}, 0, 1)$ si:
  \begin{itemize}
    \item $L \subseteq L^{\prime}$
    \item $0 = 0^{\prime}$ y $1 = 1^{\prime}$
    \item $\SU = \SU^{\prime}\mathrm{\mid}_{L \times L}$, $\IN = \IN^{\prime}\mathrm{\mid}_{L \times L}$ y
      $^{c} = ^{c^{\prime}}\mathrm{\mid}_{L}$
  \end{itemize}.

  \PN Un conjunto $\emptyset \neq S \subseteq L$ será llamado \textbf{subuniverso} de $(L, \SU, \IN, ^{c}, 0, 1)$ si es
  cerrado bajo las operaciones $\SU$, $\IN$ y $^{c}$, es decir, $x \ \SU \ y, x \ \IN \ y, x^{c} \in S$. Es fácil notar
  si $S$ es subuniverso de $(L, \SU, \IN, , ^{c}, 0, 1)$ entonces $(S, \SU\mathrm{\mid}_{S \times S}, \IN\mathrm{\mid}_{
  S \times S}, , ^{c}\mathrm{\mid}_{S}, 0, 1)$ es \textbf{subreticulado} de $(L, \SU, \IN, ^{c}, 0, 1)$.

  \vspace{3mm}
  \PN \textbf{HOMOMORFISMO E ISOMORFISMO DE RETICULADOS COMPLEMENTADOS:} Sean $(L, \SU, \IN, ^{c}, 0, 1)$ y
  $(L^{\prime}, \SU^{\prime}, \IN^{\prime}, ^{c^{\prime}}, 0^{\prime}, 1^{\prime})$ reticulados complementados.
  \begin{itemize}
    \item Una función $F: L \rightarrow L^{\prime}$ será llamada un \textbf{homomorfismo} de $(L, \SU, \IN, 0, 1)$ en
      $(L^{\prime}, \SU^{\prime}, \IN^{\prime}, 0^{\prime}, 1^{\prime})$ si $\forall x, y \in L$ se cumple que:
      \[
        \begin{array}{rcl}
          F(x \mathsf{\;s\;} y) &=& F(x) \ \SU^{\prime} \ F(y) \\
          F(x \mathsf{\;i\;} y) &=& F(x) \ \IN^{\prime} \ F(y) \\
          F(x^{c}) &=& F(x)^{c^{\prime}} \\
          F(0) &=& 0^{\prime} \\
          F(1) &=& 1^{\prime}
        \end{array}
      \]
    \item Una función $F: L \rightarrow L^{\prime}$ será llamada un \textbf{isomorfismo} de $(L, \SU, \IN, ^{c}, 0, 1)$
      en $(L^{\prime}, \SU^{\prime}, \IN^{\prime}, ^{c^{\prime}}, 0^{\prime}, 1^{\prime})$ si $F$ es \textbf{biyectiva}
      y tanto $F$ como $F^{-1}$ son \textbf{homomorfismos}.
  \end{itemize}

  \vspace{3mm}
  \PN \textbf{CONGRUENCIAS DE RETICULADOS COMPLEMENTADO:} Sea $(L, \SU, \IN, ^{c}, 0, 1)$ un reticulado, una
  \textbf{congruencia} sobre $(L, \SU, \IN, ^{c}, 0, 1)$ será una \textbf{relación de equivalencia} $\theta$ la cual
  cumpla:
  \begin{enumerate}
    \item $x \theta x^{\prime} \text{ y } y \theta y^{\prime} \Rightarrow (x \ \SU \ y) \theta (x^{\prime} \ \SU \
      y^{\prime}) \text{ y } (x \ \IN \ y) \theta (x^{\prime} \ \IN \ y^{\prime})$
    \item $x/\theta = y/\theta \Rightarrow x^{c}/\theta = y^{c}/\theta$
  \end{enumerate}

  \PN Definimos, sobre $L/\theta$, $\mathsf{\tilde{s}}, \mathsf{\tilde{i}}$ y $\mathsf{\tilde{^{c}}}$, de la siguiente
  manera:
  \[
    \begin{array}{rcl}
      x/\theta \ \mathsf{\tilde{s}} \ y/\theta &=& (x \ \SU \ y)/\theta \\
      x/\theta \ \mathsf{\tilde{i}} \ y/\theta &=& (x \ \IN \ y)/\theta \\
      (x/\theta)^{\mathsf{\tilde{^{c}}}} &=& x^{c}/\theta \\
    \end{array}
  \]

	\section{Reticulados Distributivos}

  \PN \textbf{RETICULADO DISTRIBUTIVO:} Un reticulado se llamará \textbf{distributivo} cuando cumpla alguna de las
  siguiente propiedades, para $x, y, z \in L$ cualquieras:
  \begin{enumerate}
    \item $x \ \IN \ (y \ \SU \ z) = (x \ \IN \ y) \ \SU \ (x \ \IN \ z)$
    \item $x \ \SU \ (y \ \IN \ z) = (x \ \SU \ y) \ \IN \ (x \ \SU \ z)$
  \end{enumerate}

  \vspace{3mm}
  \PN \textbf{FILTRO:} Un \textbf{filtro} de un reticulado $(L,\SU,\IN)$ será un subconjunto
  $F \subseteq L$ tal que:
  \begin{enumerate}
    \item $F\neq \emptyset $
    \item $x,y\in F\Rightarrow x\;\mathsf{i\;}y\in F$
    \item $x\in F$ y $x\leq y\Rightarrow y\in F$.
  \end{enumerate}

  \vspace{3mm}
  \PN \textbf{FILTRO GENERADO:} Dado un conjunto $S \subseteq L$, el \textbf{filtro generado por S} será el siguiente
  conjunto:
  \[
    [S) = \{y \in L: y \geq s_{1} \ \IN \dotsc \ \IN \ s_{n} \text{, para algunos } s_{1}, \dotsc, s_{n} \in S, n \geq
    1\}
  \]

  \vspace{3mm}
  \PN \textbf{FILTRO PRIMO:} Un filtro $F$ de un reticulado $(L,\SU,\IN)$ será llamado \textbf{primo}
  cuando se cumplan:
  \begin{enumerate}
    \item $F\neq L$
    \item $x\;\mathsf{s\;}y\in F\Rightarrow x\in F$ o $y\in F$.
  \end{enumerate}

  \vspace{3mm}
  \PN \textbf{ÁLGEBRA DE BOOLE:} Un \textbf{Álgebra de Boole} será un reticulado complementado y distributivo.


\end{document}
