\documentclass[12pt,a4paper]{article}
\usepackage[utf8]{inputenc}[spanish]
\usepackage{amsmath}
\usepackage{amsfonts}
\usepackage{amssymb}
\usepackage{lmodern}
\usepackage{amsmath}
\usepackage{amsthm}
\usepackage{enumerate}
\usepackage{graphicx}
\usepackage{mathtools}
\usepackage{stackrel}
\usepackage[left=2cm,right=2cm,top=2cm,bottom=2cm]{geometry}
\newcounter{neq}
\providecommand{\abs}[1]{\lvert#1\rvert}
\newcommand{\SIGMA}{\Sigma^{\ast}}
\newcommand{\PN}{\par\noindent}
\newcommand{\SU}{\mathsf{s}}
\newcommand{\IN}{\mathsf{i}}

\begin{document}
	\begin{center}
		\Huge \textbf{Definiciones de Lógica 2017} \\
		\vspace{3mm}
		\large Agustín Curto
	\end{center}
	
	\part{Estructuras algebráicas ordenadas}
		\input{1/1_posets}
		\input{1/2_reticulados}
		\input{1/3_reticulados_acotados}
		\input{1/4_reticulados_complementados}
		\input{1/5_reticulados_distributivos}
	
	\part{Términos y fórmulas}
	
	\part{Estructuras}
	
	\part{Teorias de primer orden}
	
	\part{Aritmética de Peano}

\end{document}
